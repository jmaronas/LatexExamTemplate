\documentclass[12pt,a4paper]{article}
\usepackage[]{examen}

% ===== Configuración del examen =====
\setexamtitle{Poda del Rosal}
\setexamtype{Examen Final Ordinaria}
\setdegree{Doble Grado En Jardinería Ecológica e Informática}
\setexamdate{8 de Enero de 2026}
\setexamduration{2 horas}
\setexaminstructions{
\begin{enumerate}
  \item Esterilice correctamente sus tijeras.
  \item Solo está permitido el uso de sistema operativo Linux.
  \item La realización del examen en software propietario como Matlab implica la calificación con un 0.
  \item Utilice programadores de goteo de baja presión.
\end{enumerate}
}

\begin{document}

\makeexamheader

\section{Selección de ramas}

En este apartado vamos a ver como seleccionar y podar ramas, asi como programar nuestro Arduino para que riegue nuestros rosales.

\subsection{Que hacer si se te rompen las tijeras? \nota{1p}}
¿Indique que haría si se le rompen las tijeras en medio de la poda del rosal?

\solution{Comprarme otras}

\subsubsection{Programacion del goteo a pilas con arduino\nota{0.5p}}

Saque su soldador de estaño y conecte los pines de propósito general de su Arduino al programador.

\answerbox{6}

\answerbox[\textit{respuesta continua...}]{4}

\answerbox{4}[Utilice este link para comprar unas tijeras si lo estima oportuno: www.tijeras.com]

\answerbox[\textit{respuesta continua...}]{4}[sino lo tiene claro emplee este otro link www.tijeras2.com]


\end{document}
